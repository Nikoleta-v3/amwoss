\chapter[Markov Chains]{Markov Chains}

% Introduction
\chapterinitial{M}{any} real world situations have some level of
unpredictability through randomness: the flip of a coin, the number of orders of
coffee in a shop, the winning numbers of the lottery. However, mathematics can
in fact let us make predictions about what we expect to happen. One tool used to
understand randomness is Markov chains, an area of mathematics sitting at the
intersection of probability and linear algebra.

\section{Problem}\label{sec:problem}

Consider a barber shop. The shop owners have noticed that customers will not
wait if there is no room in their waiting room and will choose to take their
business elsewhere. The Barber shop would like to make an investment so as to
avoid this situation. They know the following information:

\begin{itemize}
    \item They currently have 2 barber chairs (and 2 barbers).
    \item They have waiting room for 4 people.
    \item They usually have 10 customers arrive per hour.
    \item Each Barber takes about 15 minutes to serve a customer so they can
        serve 4 customers an hour.
\end{itemize}

This is represented diagrammatically in Figure~\ref{fig:barber-shop}.

\begin{figure}[!hbtp]
    \begin{center}
    \includestandalone[width=.6\textwidth]{./assets/barber-shop/main}
    \end{center}
    \caption{Diagrammatic representation of the barber shop as a queuing system.}
    \label{fig:barber-shop}
\end{figure}


They are planning on
reconfiguring space to either have 2 extra waiting chairs or
another barber's chair and barber.

The mathematical tool used to model this situation is a Markov chain.

\section{Theory}\label{sec:theory}

A Markov chain is a model of a sequence of random events that is defined by a
collection of \textbf{states} and rules that define how to move between these
states.

For example, in the barber shop a single number is sufficient to describe the
status of the shop. If that number is 1 this implies that 1 customer is
currently having their hair cut. If that number is 5 this implies that 2
customers are being served and 3 are waiting. The entire state space is, in this
case a finite set of integers from 0 to 6. If the system is full (all barbers
and waiting room occupied) then we are in state 6 and if there is no one at the
shop then we are in state 0. This is denoted mathematically as:

\begin{equation}
    S = \{0, 1, 2, 3, 4, 5, 6\}
    \label{eqn:barber_shop_state_space}
\end{equation}

As customers arrive and leave the system goes between states as shown in
Figure~\ref{fig:barber-shop-birth-death-no-rates}.

\begin{figure}[!hbtp]
    \begin{center}
    \includestandalone[width=.6\textwidth]{./assets/barber-shop-birth-death-no-rates/main}
    \end{center}
    \caption{Diagrammatic representation of the state space}
    \label{fig:barber-shop-birth-death-no-rates}
\end{figure}

The rules that govern how to move between these states can be defined in two
ways:

\begin{itemize}
    \item Using probabilities of changing state (or not) in a well defined time
        period.  This is called a discrete Markov chain.
    \item Using rates of change from one state to another. This is called a
        continuous time Markov chain.
\end{itemize}

For our barber shop we will consider it as a continuous Markov chain as shown
in Figure~\ref{fig:barber-shop-continuous-markov-process}

\begin{figure}[!hbtp]
    \begin{center}
    \includestandalone[width=.6\textwidth]{./assets/barber-shop-continuous-markov-process/main}
    \end{center}
    \caption{Diagrammatic representation of the state space and the transition
    rates}
    \label{fig:barber-shop-continuous-markov-process}
\end{figure}

Note that a Markov chain assumes the rates follow an exponential distribution.
One interesting property of this distribution is that it is considered
memoryless which means that if a customer has been having their hair cut for 5
minutes this does not change the rate at which their service ends.
This distribution is quite common in the real world and therefore a common
assumption.

These states and rates can be represented mathematically using a transition
matrix \(Q\) where \(Q_{ij}\) represents the rate of going from state \(i\) to
state \(j\). In this case we have:

\begin{equation}
Q =
\begin{pmatrix}
-10 &  10 &   0 &   0 &   0 &   0 &   0\\
  4 & -14 &  10 &   0 &   0 &   0 &   0\\
  0 &   8 & -18 &  10 &   0 &   0 &   0\\
  0 &   0 &   8 & -18 &  10 &   0 &   0\\
  0 &   0 &   0 &   8 & -18 &  10 &   0\\
  0 &   0 &   0 &   0 &   8 & -18 &  10\\
  0 &   0 &   0 &   0 &   0 &   8 &  -8\\
 \end{pmatrix}
 \label{eqn:barber_shop_transition_matrix}
\end{equation}

You will see that \(Q_{ii}\) are negative and ensure the rows of \(Q\) sum to 0.
This gives the total rate of change leaving state \(i\).

We can use \(Q\) to understand the probability of being in a given state after
\(t\) time unis. This is can be represented mathematically using a matrix
\(P_{t}\) where \((P_{t})_{ij}\) is the probability of being in state \(j\)
after \(t\) time units having started in state \(i\). We can use \(Q\) to
calculate \(P_{t}\) using the matrix exponential:

\begin{equation}
    P_t = e^{Qt}
    \label{eqn:continuous_time_markov_process_matrix_exponential}
\end{equation}

What is also useful is understanding the long run behaviour of the
system. This allows us to answer questions such as ``what state are we most
likely to be in on average?'' or ``what is the probability of being in the last
state on average?''.

This long run probability distribution over the state can be represented using a
vector \(\pi\) where \(\pi_i\) represents the probability of being in state
\(i\). This vector is in fact the solution to the following matrix equation:

\begin{equation}
    \pi Q = 0
    \label{eqn:continuous_time_markov_process_steady_state}
\end{equation}

In the upcoming sections we will demonstrate all of the above concepts.

\section{Solving with Python}\label{sec:solving-with-python}

The first step we will take is to write a function to obtain the transition
rates between two given states:


\begin{pyin}
def get_transition_rate(
    in_state, out_state, waiting_room=4, num_barbers=2,
):
    """Return the transition rate for two given states.

    Args:
        in_state: an integer
        out_state: an integer
        waiting_room: an integer (default: 4)
        num_barbers:  an integer (default: 2)

    Returns:
        A real.
    """
    arrival_rate = 10
    service_rate = 4

    capacity = waiting_room + num_barbers
    delta = out_state - in_state

    if delta == 1 and in_state < capacity:
        return arrival_rate

    if delta == -1:
        return min(in_state, num_barbers) * service_rate

    return 0
\end{pyin}

Next, we write a function that creates an entire transition rate matrix \(Q\)
for a given problem. We will use the \mintinline{python}{numpy} to handle all
the linear algebra and the \mintinline{python}{itertools} library for some
iterations:

\begin{pyin}
import itertools
import numpy as np


def get_transition_rate_matrix(waiting_room=4, num_barbers=2):
    """Return the transition matrix Q.

    Args:
        waiting_room: an integer (default: 4)
        num_barbers: an integer (default: 2)

    Returns:
        A matrix.
    """
    capacity = waiting_room + num_barbers
    state_pairs = itertools.product(
        range(capacity + 1), repeat=2
    )

    flat_transition_rates = [
        get_transition_rate(
            in_state=in_state,
            out_state=out_state,
            waiting_room=waiting_room,
            num_barbers=num_barbers,
        )
        for in_state, out_state in state_pairs
    ]
    transition_rates = np.reshape(
        flat_transition_rates, (capacity + 1, capacity + 1)
    )
    np.fill_diagonal(
        transition_rates, -transition_rates.sum(axis=1)
    )

    return transition_rates
\end{pyin}

Using this we can obtain the matrix \(Q\) for our default system:

\begin{pyin}
Q = get_transition_rate_matrix()
print(Q)
\end{pyin}

which gives:

\begin{pyout}
[[-10  10   0   0   0   0   0]
 [  4 -14  10   0   0   0   0]
 [  0   8 -18  10   0   0   0]
 [  0   0   8 -18  10   0   0]
 [  0   0   0   8 -18  10   0]
 [  0   0   0   0   8 -18  10]
 [  0   0   0   0   0   8  -8]]
\end{pyout}

We can take the matrix exponential as
discussed above. To do this, we need to use the \mintinline{python}{scipy}
library. To see what would happen after .5 time units we obtain:
\begin{pyin}
import scipy.linalg

print(scipy.linalg.expm(Q * 0.5).round(5))
\end{pyin}

which gives:

\begin{pyout}
[[0.10492 0.21254 0.20377 0.17142 0.13021 0.09564 0.0815 ]
 [0.08501 0.18292 0.18666 0.1708  0.14377 0.1189  0.11194]
 [0.06521 0.14933 0.16338 0.16478 0.15633 0.14751 0.15346]
 [0.04388 0.10931 0.13183 0.15181 0.16777 0.18398 0.21142]
 [0.02667 0.07361 0.10005 0.13422 0.17393 0.2189  0.27262]
 [0.01567 0.0487  0.07552 0.11775 0.17512 0.24484 0.32239]
 [0.01068 0.03668 0.06286 0.10824 0.17448 0.25791 0.34914]]
\end{pyout}

To see what would happen after 500 time units we obtain:

\begin{pyin}
print(scipy.linalg.expm(Q * 500).round(5))
\end{pyin}

which gives:

\begin{pyout}
[[0.03431 0.08577 0.10722 0.13402 0.16752 0.2094  0.26176]
 [0.03431 0.08577 0.10722 0.13402 0.16752 0.2094  0.26176]
 [0.03431 0.08577 0.10722 0.13402 0.16752 0.2094  0.26176]
 [0.03431 0.08577 0.10722 0.13402 0.16752 0.2094  0.26176]
 [0.03431 0.08577 0.10722 0.13402 0.16752 0.2094  0.26176]
 [0.03431 0.08577 0.10722 0.13402 0.16752 0.2094  0.26176]
 [0.03431 0.08577 0.10722 0.13402 0.16752 0.2094  0.26176]]
\end{pyout}

We see that no matter what state (row) the system is in, after 500 time units
the probabilities are all the same. We could in fact stop our analysis here,
however our choice of 500 time units was arbitrary and might not be the correct
amount for all possible scenarios, as such we will continue to aim to solve the
underlying equation~\ref{eqn:continuous_time_markov_process_steady_state}
directly.

To do this we will solve the underlying system using a numerically efficient
algorithm called least squares optimisation (available from the
\mintinline{python}{numpy} library):

\begin{pyin}
def get_steady_state_vector(Q):
    """Return the steady state vector of any given continuous
    time transition rate matrix.

    Args:
       Q: a transition rate matrix

    Returns:
        A vector
    """
    state_space_size, _ = Q.shape
    A = np.vstack((Q.T, np.ones(state_space_size)))
    b = np.append(np.zeros(state_space_size), 1)
    x, _, _, _ = np.linalg.lstsq(A, b, rcond=None)
    return x
\end{pyin}

So if we now see the steady state vector for our default system:

\begin{pyin}
print(get_steady_state_vector(Q).round(5))
\end{pyin}

we get:

\begin{pyout}
[0.03431 0.08577 0.10722 0.13402 0.16752 0.2094  0.26176]
\end{pyout}

We can see that the shop is expected to be empty approximately 3.4\% of the time
and full 26.2\% of the time.

The final function we will write is one that uses all of
the above to just return the probability of the shop being full.

\begin{pyin}
def get_probability_of_full_shop(
    waiting_room=4, num_barbers=2
):
    """Return the probability of the barber shop being full.

    Args:
        waiting_room: an integer (default: 4)
        num_barbers: an integer (default: 2)

    Returns:
        A real.
    """
    Q = get_transition_rate_matrix(
        waiting_room=waiting_room, num_barbers=num_barbers,
    )
    pi = get_steady_state_vector(Q)
    return pi[-1]
\end{pyin}

We can now confirm the previous probability calculated probability of the shop
being full:

\begin{pyin}
print(round(get_probability_of_full_shop(), 6))
\end{pyin}

which gives:

\begin{pyout}
0.261756
\end{pyout}

If we were too have 2 extra space in the waiting room:

\begin{pyin}
print(round(get_probability_of_full_shop(waiting_room=6), 6))
\end{pyin}

which gives:

\begin{pyout}
0.23557
\end{pyout}

This is a slight improvement however, increasing the number of barbers has a
substantial effect:

\begin{pyin}
print(round(get_probability_of_full_shop(num_barbers=3), 6))
\end{pyin}

\begin{pyout}
0.078636
\end{pyout}
\section{Solving with R}\label{sec:solving-with-R}

The first step we will take is write a function to obtain the transition rates
between two given states:

\begin{Rin}
#' Return the transition rate for two given states.
#'
#' @param in_state an integer
#' @param out_state an integer
#' @param waiting_room an integer (default: 4)
#' @param num_barbers an integer  (default: 2)
#'
#' @return A real
get_transition_rate <- function(in_state,
                                out_state,
                                waiting_room = 4,
                                num_barbers = 2){
  arrival_rate <- 10
  service_rate <- 4

  capacity <- waiting_room + num_barbers
  delta <- out_state - in_state

  if (delta == 1) {
    if (in_state < capacity) {
      return(arrival_rate)
    }
  }

  if (delta == -1) {
    return(min(in_state, num_barbers) * service_rate)
  }
  return(0)
}
\end{Rin}

We will not actually use this function but a vectorized version of this:

\begin{Rin}
vectorized_get_transition_rate <- Vectorize(
  get_transition_rate,
  vectorize.args = c("in_state", "out_state")
)
\end{Rin}

This function can now take a vector of inputs for the \mintinline{R}{in_state}
and \mintinline{R}{out_state} variables which will allow us to simplify the
following code that creates the matrices:

\begin{Rin}
#' Return the transition rate matrix Q
#'
#' @param waiting_room an integer (default: 4)
#' @param num_barbers an integer (default: 2)
#'
#' @return A matrix
get_transition_rate_matrix <- function(waiting_room = 4,
                                       num_barbers = 2){
  max_state <- waiting_room + num_barbers

  Q <- outer(0:max_state,
    0:max_state,
    vectorized_get_transition_rate,
    waiting_room = waiting_room,
    num_barbers = num_barbers
  )
  row_sums <- rowSums(Q)

  diag(Q) <- -row_sums
  Q
}
\end{Rin}

Using this we can obtain the matrix \(Q\) for our default system:

\begin{Rin}
Q <- get_transition_rate_matrix()
print(Q)
\end{Rin}

which gives:

\begin{Rout}
     [,1] [,2] [,3] [,4] [,5] [,6] [,7]
[1,]  -10   10    0    0    0    0    0
[2,]    4  -14   10    0    0    0    0
[3,]    0    8  -18   10    0    0    0
[4,]    0    0    8  -18   10    0    0
[5,]    0    0    0    8  -18   10    0
[6,]    0    0    0    0    8  -18   10
[7,]    0    0    0    0    0    8   -8
\end{Rout}

One immediate thing we can do with this matrix is take the matrix exponential
discussed above. To do this, we need to use an R library call
\mintinline{R}{expm}.

To be able to make use of the nice \mintinline{R} "pipe" operator we are
also going to load the \mintinline{R}{magrittr} library:

Now if we wanted to see what would happen after .5 time units we obtain:

\begin{Rin}
library(expm, warn.conflicts = FALSE, quietly = TRUE)
library(magrittr, warn.conflicts = FALSE, quietly = TRUE)

print( (Q * .5) %>% expm %>% round(5))
\end{Rin}

which gives:

\begin{Rout}
        [,1]    [,2]    [,3]    [,4]    [,5]    [,6]    [,7]
[1,] 0.10492 0.21254 0.20377 0.17142 0.13021 0.09564 0.08150
[2,] 0.08501 0.18292 0.18666 0.17080 0.14377 0.11890 0.11194
[3,] 0.06521 0.14933 0.16338 0.16478 0.15633 0.14751 0.15346
[4,] 0.04388 0.10931 0.13183 0.15181 0.16777 0.18398 0.21142
[5,] 0.02667 0.07361 0.10005 0.13422 0.17393 0.21890 0.27262
[6,] 0.01567 0.04870 0.07552 0.11775 0.17512 0.24484 0.32239
[7,] 0.01068 0.03668 0.06286 0.10824 0.17448 0.25791 0.34914
\end{Rout}

After 500 time units we obtain:

\begin{Rin}
print( (Q * 500) %>% expm %>% round(5))
\end{Rin}

which gives:

\begin{Rout}
        [,1]    [,2]    [,3]    [,4]    [,5]   [,6]    [,7]
[1,] 0.03431 0.08577 0.10722 0.13402 0.16752 0.2094 0.26176
[2,] 0.03431 0.08577 0.10722 0.13402 0.16752 0.2094 0.26176
[3,] 0.03431 0.08577 0.10722 0.13402 0.16752 0.2094 0.26176
[4,] 0.03431 0.08577 0.10722 0.13402 0.16752 0.2094 0.26176
[5,] 0.03431 0.08577 0.10722 0.13402 0.16752 0.2094 0.26176
[6,] 0.03431 0.08577 0.10722 0.13402 0.16752 0.2094 0.26176
[7,] 0.03431 0.08577 0.10722 0.13402 0.16752 0.2094 0.26176
\end{Rout}

We see that no matter what state (row) we are in, after 500 time units the
probabilities are all the same. We could in fact stop our analysis here, however
our choice of 500 time units was arbitrary and might not be the correct amount
for all possible scenarios, as such we will continue to aim to solve the
underlying equation~\ref{eqn:continuous_time_markov_process_steady_state}
directly.

To be able to do this, we will make use of the versatile \mintinline{R}{pracma}
package which includes a number of numerical analysis functions for efficient
computations.

\begin{Rin}
library(pracma, warn.conflicts = FALSE, quietly = TRUE)

#' Return the steady state vector of any given continuous time
#' transition rate matrix
#'
#' @param Q a transition rate matrix
#'
#' @return A vector
get_steady_state_vector <- function(Q){
  state_space_size <- dim(Q)[1]
  A <- rbind(t(Q), 1)
  b <- c(integer(state_space_size), 1)
  mldivide(A, b)
}
\end{Rin}

This is making use of \mintinline{R}{pracma}'s \mintinline{R}{mldivide} function
which chooses the best numerical algorithm to find the solution to a given
matrix equation \(Ax=b\).

So if we now see the steady state vector for our default system:

\begin{Rin}
print(get_steady_state_vector(Q))
\end{Rin}

we get:

\begin{Rout}
           [,1]
[1,] 0.03430888
[2,] 0.08577220
[3,] 0.10721525
[4,] 0.13401906
[5,] 0.16752383
[6,] 0.20940479
[7,] 0.26175598
\end{Rout}

We can see that the shop is expected to be empty approximately 3.4\% of the time
and full 26.2\% of the time.

The final piece of this puzzle is to create a single function that uses all of
the above to just return the probability of the shop being full.

\begin{Rin}
#' Return the probability of the barber shop being full
#'
#' @param waiting_room (default: 4)
#' @param num_barbers (default: 2)
#'
#' @return A real
get_probability_of_full_shop <- function(waiting_room = 4,
                                         num_barbers = 2){
  arrival_rate <- 10
  service_rate <- 4
  pi <- get_transition_rate_matrix(
    waiting_room = waiting_room,
    num_barbers = num_barbers
    ) %>%
    get_steady_state_vector()

  capacity <- waiting_room + num_barbers
  pi[capacity + 1]
}
\end{Rin}

Now we can run this code efficiently with both scenarios:

\begin{Rin}
print(get_probability_of_full_shop(waiting_room = 6))
\end{Rin}

which decreases the probability of a full shop to:

\begin{Rout}
[1] 0.2355699
\end{Rout}

but adding another barber and chair:

\begin{Rin}
print(get_probability_of_full_shop(num_barbers = 3))
\end{Rin}

gives:

\begin{Rout}
[1] 0.0786359
\end{Rout}

\section{Research}\label{sec:research}

TBA
