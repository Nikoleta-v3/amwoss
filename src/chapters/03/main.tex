\chapter[Markov Chains]{Markov Chains}

% Introduction
\chapterinitial{M}{any} real world situations have some level of
unpredictability through randomness: the flip of a coin, the number of orders of
coffee in a shop, the winning numbers of the lottery. However, mathematics can
in fact let us make predictions about what we expect to happen. One tool used to
understand randomness is Markov chains, an area of mathematics sitting at the
intersection of probability and linear algebra.

\section{Problem}\label{sec:problem}

- Barber shop
- 2 chairs and 2 barbers
- Room for 4 people to wait
- Add room for 2 people to wait or add 1 more barber + chair?
- What is the probability of having to wait?

\section{Theory}\label{sec:theory}

- System can be described using a Markov chain which is a collection of States,
transitions etc...
- In this instance, it is a type of birth-death process that can be drawn...
- Has a Matrix representation: Q
- We can calculate the expected smallest time for a change to occur.
- There is a mathematical process that discretises this system to give us P and
the corresponding picture...
- In Delta t time what is the probability...
- Now we can think about P pi, ..., P to a high power will give us...  this will
correspond to pi P = pi (this is an eigenvalue problem).
- Return to continuous system.

\section{Solving with Python}\label{sec:solving-with-python}

- numpy array
- function to generate Q
- function to discretise Q
- multiple P by vectors
- raise P to high power
- write function to solve matrix equation and return pfull - eigenvalue approach

- Run for both scenarios


\section{Solving with R}\label{sec:solving-with-R}

The first step we will take is write a function to obtain the transition rates
between two given states:

\begin{Rin}
#' Return the transition rate for two given states.
#'
#' @description
#' For a given barber shop problem this returns the transition rate between two
#' states. As well as taking the two states as inputs it takes a number of
#' parameters that define the system.
#'
#' @param in_state an integer denoting the current state
#' @param out_state an integer denoting the next state
#' @param waiting_room an integer denoting the size of the waiting room
#' (default: 4)
#' @param number_of_barbers an integer denoting the number of barber and chairs
#' (default: 2)
#' @param arrival_rate a real number denoting the number of individuals per
#' unit time that arrive at the barber shop (default: 10)
#' @param service_rate a real number denoting the number of individuals per unit
#' time that a single barber can serve (default: 4)
#'
#' @return A real
get_transition_rate <- function(
                                in_state,
                                out_state,
                                waiting_room = 4,
                                number_of_barbers = 2,
                                arrival_rate = 10,
                                service_rate = 4) {
  capacity <- waiting_room + number_of_barbers
  delta <- out_state - in_state

  if (delta == 1) {
    if (in_state < capacity) {
      return(arrival_rate)
    }
  }

  if (delta == -1) {
    return(min(in_state, number_of_barbers) * service_rate)
  }
  return(0)
}
\end{Rin}

We will not actually use this function but a vectorized version of this:

\begin{Rin}
vectorized_get_transition_rate <- Vectorize(
  get_transition_rate,
  vectorize.args = c("in_state", "out_state")
)
\end{Rin}

This function can now take a vector of inputs for the \mintinline{R}{in_state}
and \mintinline{R}{out_state} variables which will allow us to simplify the
following code that creates the matrices:

\begin{Rin}
#' Return the transition rate matrix Q
#'
#' @description
#' For a given barber shop problem this returns the transition rate matrix
#' As inputs it takes a number of
#' parameters that define the system.
#'
#' @param waiting_room an integer denoting the size of the waiting room
#' (default: 4)
#' @param number_of_barbers an integer denoting the number of barber and chairs
#' (default: 2)
#' @param arrival_rate a real number denoting the number of individuals per
#' unit time that arrive at the barber shop (default: 10)
#' @param service_rate a real number denoting the number of individuals per unit
#' time that a single barber can serve (default: 4)
#'
#' @return A matrix
get_transition_rate_matrix <- function(
                                       waiting_room = 4,
                                       number_of_barbers = 2,
                                       arrival_rate = 10,
                                       service_rate = 4) {
  max_state <- waiting_room + number_of_barbers

  Q <- outer(0:max_state,
    0:max_state,
    vectorized_get_transition_rate,
    waiting_room = waiting_room,
    number_of_barbers = number_of_barbers,
    arrival_rate = arrival_rate,
    service_rate = service_rate
  )
  row_sums <- rowSums(Q)

  diag(Q) <- -row_sums
  Q
}
\end{Rin}

Using this we can obtain the matrix \(Q\) for our default system:

\begin{Rin}
Q <- get_transition_rate_matrix()
Q
\end{Rin}

which gives:

\begin{Rout}
     [,1] [,2] [,3] [,4] [,5] [,6] [,7]
[1,]  -10   10    0    0    0    0    0
[2,]    4  -14   10    0    0    0    0
[3,]    0    8  -18   10    0    0    0
[4,]    0    0    8  -18   10    0    0
[5,]    0    0    0    8  -18   10    0
[6,]    0    0    0    0    8  -18   10
[7,]    0    0    0    0    0    8   -8
\end{Rout}

One immediate thing we can do with this matrix is take the matrix exponential
discussed above. To do this, we need to use an R library call
\mintinline{R}{expm}:

\begin{Rin}
library(expm, warn.conflicts = FALSE, quietly = TRUE)
\end{Rin}

To be able to make use of the nice \mintinline{R} "pipe" operator we are
also going to load the \mintinline{R}{dplyr} library:

\begin{Rin}
library(dplyr, warn.conflicts = FALSE, quietly = TRUE)
\end{Rin}

Now if we wanted to see what would happen after 500 time units we obtain:

\begin{Rin}
(t(Q) * 500) %>% expm %>% round(6)
\end{Rin}

which gives:

\begin{Rout}
         [,1]     [,2]     [,3]     [,4]     [,5]     [,6]     [,7]
[1,] 0.034309 0.034309 0.034309 0.034309 0.034309 0.034309 0.034309
[2,] 0.085772 0.085772 0.085772 0.085772 0.085772 0.085772 0.085772
[3,] 0.107215 0.107215 0.107215 0.107215 0.107215 0.107215 0.107215
[4,] 0.134019 0.134019 0.134019 0.134019 0.134019 0.134019 0.134019
[5,] 0.167524 0.167524 0.167524 0.167524 0.167524 0.167524 0.167524
[6,] 0.209405 0.209405 0.209405 0.209405 0.209405 0.209405 0.209405
[7,] 0.261756 0.261756 0.261756 0.261756 0.261756 0.261756 0.261756
\end{Rout}

We see that no matter what state (column) we are in, after 500 time units the
probabilities are all the same. We could in fact stop our analysis here, however
our choice of 500 time units was arbitrary and might not be the correct amount
for all possible scenarios, as such we will continue to aim to solve the
underlying equation directly. % TODO Point at underlying equation.

To be able to do this, we will make use of the versatile \mintinline{R}{pracma}
package which includes a number of numerical analysis functions for efficient
computations.

\begin{Rin}
library(pracma, warn.conflicts = FALSE, quietly = TRUE)

#' Return the steady state vector of any given continuous time transition rate
#' matrix
#'
#' @description
#' Solves the matrix equation Ax=b where A is the matrix Q with an extra row of
#' ones (to ensure a probability vector is given) and b is a vector of 0s and a
#' single one.
#'
#' @param Q a transition rate matrix
#'
#' @return A vector
get_steady_state_vector <- function(Q){
  state_space_size <- dim(Q)[1]
  A <- rbind(t(Q), 1)
  b <- c(integer(state_space_size), 1)
  mldivide(A, b)
}
\end{Rin}

This is making use of \mintinline{R}{pracma}'s \mintinline{mldivide} function
which chooses the best numerical algorithm to find the solution to a given
matrix equation \(Ax=b\).

So if we now see the steady state vector for our default system:

\begin{Rin}
get_steady_state_vector(Q)
\end{Rin}

we get:

\begin{Rout}
           [,1]
[1,] 0.03430888
[2,] 0.08577220
[3,] 0.10721525
[4,] 0.13401906
[5,] 0.16752383
[6,] 0.20940479
[7,] 0.26175598
\end{Rout}

We can see that the shop is expected to be empty approximately 3.4\% of the time
and full 26.2\% of the time.

The final piece of this puzzle is to create a single function that uses all of
the above to just return the probability of the shop being full.

\begin{Rin}
#' Return the probability of the barber shop being full
#'
#' @description
#' For a given barber shop problem this returns the probability of it being full
#' As inputs it takes a number of
#' parameters that define the system.
#'
#' @param waiting_room an integer denoting the size of the waiting room
#' (default: 4)
#' @param number_of_barbers an integer denoting the number of barber and chairs
#' (default: 2)
#' @param arrival_rate a real number denoting the number of individuals per
#' unit time that arrive at the barber shop (default: 10)
#' @param service_rate a real number denoting the number of individuals per unit
#' time that a single barber can serve (default: 4)
#'
#' @return A real
get_probability_of_full_shop <- function(
                                         waiting_room = 4,
                                         number_of_barbers = 2,
                                         arrival_rate = 10,
                                         service_rate = 4) {
  pi <- get_transition_rate_matrix(
    waiting_room = waiting_room,
    number_of_barbers = number_of_barbers,
    arrival_rate = arrival_rate,
    service_rate = service_rate
  ) %>%
    get_steady_state_vector()

  capacity <- waiting_room + number_of_barbers
  pi[capacity]
}
\end{Rin}

Now we can run this code efficiently with both scenarios:

\begin{Rin}
get_probability_of_full_shop(waiting_room = 6)
\end{Rin}

which decreases the probability of a full shop to:

\begin{Rout}
[1] 0.1884559
\end{Rout}

but adding another barber and chair:

\begin{Rin}
get_probability_of_full_shop(number_of_barbers = 3)
\end{Rin}

gives:

\begin{Rout}
[1] 0.09436308
\end{Rout}

In fact even with room for 20 people to wait the only way 2 barbers would be
able to have a less than 10\% of the shop being full is to find a way to each
serve .4 more of a customer per hour:

\begin{Rin}
get_probability_of_full_shop(waiting_room = 20, service_rate = 4.5)
\end{Rin}

\begin{Rout}
[1] 0.09928879
\end{Rout}

\section{Research}\label{sec:research}

TBA
