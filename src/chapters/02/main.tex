This book will involve using software, the particular interface to software we
will use is to write code. There are numerous reasons why this is the correct
way to do things but one of them is reproducibility. % TODO Include a reference here

This chapter will go over the basics of getting your computer set up to use the
software: the programming languages R % TODO include reference
and Python. % TODO include reference.
It will also briefly discuss using the command line: a particular interface to
your whole compute and finally it will give a brief introduction to R and
Python.

What this Chapter (and indeed this whole book) is not is a place to learn R and
Python completely. We will cover very specific tasks and how to carry them out
but we will not cover the entire intricacies of each language. There are
numerous places (books, websites, courses) that are available to do that.
A lot of these places would argue that you should not learn multiple
programming books from one book and instead concentrate on a single skill at a
time. We agree and the single skill to concentrate on with this book is the use
of software to solve applied mathematical problems. The software itself is not
the most important component.

\section{Software installation}\label{sec:software-installation}

There are a number of different places from which you can buy your vegetables,
you can grow them yourself, you can go to a market and pick fresh fruit from
specific stalls, you can go to a supermarket and buy a bag of a collection of
vegetables and in some places you can even get a box of vegetables regularly
posted to you. Software is similar, there are a variety of places from which you
can get it and a number of different forms in which it can be obtained.

IF you're comfortable with using R and Python then you probably do not need to
read this section and you might even use different so called ``distributions''
of each piece of software but for the purpose of this book here is where we will
be getting what we need:

\begin{itemize}
    \item Python: we will use the Anaconda distribution:
        \href{https://www.anaconda.com/distribution/}
    \item R: we will be getting this directly from the Comprehensive R Archive
        Network (commonly referred to as CRAN):
        \href{https://cran.r-project.org}. We will also use another piece of
        software called Rstudio: \href{https://rstudio.com}.
\end{itemize}

\subsection{Installing Python}\label{sec:installing-python}

Installing Python and all the software we need around it is done by downloading
and running the installer for the Anaconda distribution.

\begin{enumerate}
    \item Go to this webpage: \href{https://www.anaconda.com/download/}.
    \item Identify and download the version of Python 3 for your operating system
        (Windows, Mac OSX, Linux). Run the installer.
\end{enumerate}

\subsection{Installing R}\label{sec:installing-R}

There are actually two pieces of software we need to install to use R for the
purposes of this book, first the R language itself and second an application
with which we will write R code.

\begin{enumerate}
    \item Go to this webpage: \href{https://cran.r-project.org}.
    \item Identify and download the latest version of R for your operating system
        (Windows, Mac OSX, Linux). Run the installer.
    \item Go to this webpage: \href{https://rstudio.com}.
    \item Identify and download the latest version of Rstudio for your operating system
        (Windows, Mac OSX, Linux). Run the installer.
\end{enumerate}

\section{Using the command line}\label{sec:using-the-command-line}

There are various interfaces to using a computer, the most common one is to use
a mouse and keyboard and click on programmes we want to use. Another approach is
to use what is called a command line interface this is where we do not interact
visually with a computer but we type in specific commands.

% TODO Add image of command line on a variety of computer systems.

We can use our command line to navigate the various directories on our
computer.

There are two types of operating systems that we consider here:

\begin{itemize}
    \item Windows
    \item Nix: this includes OSX (the Mac operating system) and Linux
\end{itemize}

Not all commands are the same on each type of operating system.

So let us start by opening our command line interface:

\begin{itemize}
    \item Windows: after having installed Anaconda look to open the
        AnacondaPrompt. There are a number of other command line interfaces
        available but this is the one we recommend for the purposes of this
        book.
    \item Nix: look to open the Terminal.
\end{itemize}

This should open something that looks like % TODO add screenshot
and somewhat resembles a black box with some text in it.
This is where we will write our commands to the computer.

For example to list the contents of the directory we are currently in:

\textbf{On nix:}

\begin{minted}{bash}
ls
\end{minted}

\textbf{On Windows}

\begin{minted}{bash}
dir
\end{minted}

It is also possible to get the name of the directory we are currently in:

\textbf{On nix:}

\begin{minted}{bash}
pwd
\end{minted}

\textbf{On Windows}

\begin{minted}{bash}
cd
\end{minted}

Finally we can also use the command line to move to another directory. The
command for this are the same on Nix and on Windows.

\begin{minted}{bash}
cd <name_of_subdirectory>
\end{minted}

% TODO Include nice diagram showing cd (and include `..` in there).

The command line is an important tool to learn to use when doing tasks:

\begin{itemize}
    \item If we want to scale the tasks, a commonly heard phrase is that `mouse
        clicks do not scale' highlighting that to repeat a task many times when
        using a graphical interface is inefficient.
    \item If we want someone else to be able to repeat the tasks, we can use
        screenshots of graphical interfaces but there will always be a level of
        ambiguity whereas the commands use in the command line are precise.
\end{itemize}

We can use our two programming languages right within the command line interface
(we will actually be using a different tool that we will describe shortly).

To use Python, simply type the following and press Enter:

\begin{minted}{bash}
python
\end{minted}

This should make something like the following appear:

\begin{minted}{python}
Python 3.7.1 | packaged by conda-forge | (default, Nov 13 2018, 10:30:07)
[Clang 4.0.1 (tags/RELEASE_401/final)] :: Anaconda, Inc. on darwin
Type "help", "copyright", "credits" or "license" for more information.
>>>
\end{minted}

The \mintinline{python}{>>>} is a prompt ready to accept a Python command. Let
us start with the following:

\begin{minted}{python}
>>> 2 + 2
\end{minted}

When you press Enter, this will give:

\begin{minted}{python}
4
\end{minted}

This particular way of using Python is called a REPL which stands for: `Read
eval print loop' which indicates that it takes a command, evaluates it and waits
for the next one.

To quit Python's REPL type the following (note that \mintinline{python}{()},
more about that later):

\begin{minted}{python}
>>> quit()
\end{minted}

We can do the same for R. To start R's REPL, in your command line type the
following and press Enter:

\begin{minted}{bash}
R
\end{minted}

This should make something like the following appear:

\begin{minted}{R}
R version 3.5.1 (2018-07-02) -- "Feather Spray"
Copyright (C) 2018 The R Foundation for Statistical Computing
Platform: x86_64-apple-darwin13.4.0 (64-bit)

R is free software and comes with ABSOLUTELY NO WARRANTY.
You are welcome to redistribute it under certain conditions.
Type 'license()' or 'licence()' for distribution details.

  Natural language support but running in an English locale

R is a collaborative project with many contributors.
Type 'contributors()' for more information and
'citation()' on how to cite R or R packages in publications.

Type 'demo()' for some demos, 'help()' for on-line help, or
'help.start()' for an HTML browser interface to help.
Type 'q()' to quit R.

>
\end{minted}

The \mintinline{R}{>} is a prompt ready to accept an R command. Let
us start with the following:

\begin{minted}{R}
> 2 + 2
\end{minted}

When you press Enter, this will give:

\begin{minted}{R}
4
\end{minted}

To quit R's REPL type the following:

\begin{minted}{python}
> q()
\end{minted}

This will bring up a further prompt asking you to save some information about
what you just did. You can type \mintinline{R}{n} for now:

\begin{minted}{python}
> Save workspace image? [y/n/c]: n
\end{minted}

These two REPLs are not unique and also not the most efficient way of using the
languages however they can at times be useful if you just want to type a very
short command or perhaps check something quickly.

Another approach is to save a collection of commands in a plain text file and
pass it to the interpreter at the command line.

For example, if we had a number of Python commands in \mintinline{bash}{main.py}
we could run this at the command line using:

\begin{minted}{bash}
python main.py
\end{minted}

Similarly for a file with a number of R commands \mintinline{bash}{main.R}:

\begin{minted}{bash}
Rscript main.R
\end{minted}

These are just a few of many ways to use Python and R, an important notion to
understand is that Python and R are not the particular tools that we use to
interface to them. On a day to day basis the authors of this book will use both
of the above approaches as well as the next ones, we recommend readers take time
to experiment and understand the particular use cases for which each tool works
best for them.

The two tools we recommend to use in this book are:

\begin{itemize}
    \item For Python: the Jupyter notebook, a tool that behaves similarly to a
        REPL, runs in the web browser and is very popular in research.
    \item For R: RStudio, an integrated development environment with a lot of
        helpful features.
\end{itemize}

The best way to start the Jupyter notebook is to type the following in your
command line:

\begin{minted}{bash}
jupyter notebook
\end{minted}

This will create a \textit{notebook server} that runs on your computer and
should open a page that looks like % TODO include screen shot
Note that despite running in a web browser this does not need the internet to
run.

We can create a new notebook and write and run code in the \textit{cells}.
% TODO  maybe include a screenshot with arrows etc...

To start Rstudio, locate the application on your computer and double click on
it. This will open an application that looks like % TODO include screen shot

Rstudio includes its on REPL, so we can type and run single commands there but
we can also write in a file that we can run
% TODO include a screenshot with arrows etc...

In the next sections we will cover some basics of Python and R.

\section{Basic Python}\label{sec:basic-python}

This section gives a very brief overview of some introductory aspects of Python,
there are excellent resources available for learning Python and we recommend the
reader goes there if they feel they need an in depth understanding of the
language % TODO Cite courses/books etc...

In the previous section, we saw how to get Python to perform a single
calculation:

\begin{minted}{python}
3 + 5
\end{minted}

We can also assign values to a variable:

\begin{minted}{python}
a = 3
b = 5
c = a + b
\end{minted}

There are a number of different types of variables in Python, here is a very
brief list of some of them:

\begin{itemize}
    \item Integers -- \mintinline{python}{int} -- for example \mintinline{python}{2},
        \mintinline{python}{4}, \mintinline{python}{-459060}.
    \item Floats -- \mintinline{python}{float} -- for example \mintinline{python}{2.0},
        \mintinline{python}{3.4}, \mintinline{python}{-3.459060}.
    \item Strings -- \mintinline{python}{str} -- for example \mintinline{python}{"two"}, \mintinline{python}{"hello
        world"}, \mintinline{python}{"3450"}.
    \item Booleans -- \mintinline{python}{bool} -- for example \mintinline{python}{True} or
        \mintinline{python}{False}.
\end{itemize}

Based on the values of a variable it is possible to construct Booleans:

\begin{minted}{python}
is_a_larger_than_b = a > b
\end{minted}

The variable \mintinline{python}{is_a_larger_than_b} will be the boolean variable
\mintinline{python}{False}.

This is an important concept as boolean variable allow us to use conditional
statements that let us write code that does specific things based on the value
of variables. For example the following code will add 5 to the smallest
variable:

\begin{minted}{python}
a = 3
b = 5
if a < b:
    a = a + 3
elif a > b:
    b = b + 5
else:
    a = a + 3
    b = b + 3
\end{minted}

If you are experimenting by typing the code as you go change the value of
\mintinline{python}{a} or \mintinline{python}{b} to see how the behaviour changes.
What happens if they are equal?

It is also possible to use these conditional statements to repeat code. For
example the following code will repeatedly add 1 to the smallest variable until
it becomes equal to the largest one:

\begin{minted}{python}
a = 3
b = 5
while a != b:
    if a < b:
        a = a + 1
    else:
        b = b + 1
\end{minted}

It is important to be able to reuse code, this is done using a programming
concept called a \textit{function}, which acts similarly to a mathematical
function.

The following code, creates a function that takes two variables as input and
outputs the largest number and the smallest increased by 3.

\begin{minted}{python}
def add_3_to_smallest(a, b):
    """ 
    This function adds 3 to the smallest of a or b.
    """
    if a < b:
        return a + 3, b
    return a, b + 3
\end{minted}

Once we have defined the function, the following is how we use it:

\begin{minted}{python}
add_3_to_smallest(a=5, b=-42)
\end{minted}

Python has a type of variable that is in fact a collection of pointers to other
variables. This is called a list. Here for example is a collection of strings:

\begin{minted}{python}
tennis_players = ["Federer", "S. Williams", "V. Williams", "Nadal", "King"]
\end{minted}

There are a number of things that can be done with lists but one particular
aspect is that they are a sub type of something called an iterable in Python
which means we can iterate over them. We do this in Python using a
\mintinline{python}{for} loop. For example, the following code will iterate over
the list and print all the values:

\begin{minted}{python}
for name in tennis_players:
    print(name)
\end{minted}

We will often want to iterate over a set of integers, Python has a
\mintinline{python}{range} command that can create such a set with ease. The following
code will print every 3 integers from 30 to 50:


\begin{minted}{python}
for integer in range(30, 50, 3):
    print(integer)
\end{minted}

A final important aspect of Python is that of libraries. The code examples above
are from the so called `standard library' but Python has numerous libraries
specific to given problems. A lot of these libraries came bundled with the
anaconda distribution but if you want to download one that is not you can always
do so as long as you have an internet connection.

For example, to download a library for studying queueing systems
\mintinline{python}{ciw} open your command line interface and type the
following:

\begin{minted}{bash}
pip install ciw
\end{minted}

Once you restart your python interpreter, for exampl if you are using a Jupyter
notebook then restart the Kernel and then you can run the following to make
\mintinline{python}{ciw} available to you:

\begin{minted}{python}
import ciw
\end{minted}

\section{Basic R}\label{sec:basic-R}

This section gives a very brief overview of some introductory aspects of R,
there are excellent resources available for learning R and we recommend the
reader goes there if they feel they need an in depth understanding of the
language % TODO Cite courses/books etc...

In the previous section, we saw how to get R to perform a single
calculation:

\begin{minted}{R}
3 + 5
\end{minted}

We can also assign values to a variable:

\begin{minted}{R}
a <- 3
b <- 5
c <- a + b
\end{minted}

An important difference between R and Python is that in R the base structure is
in fact a vector, even if it only contains a single variable. We can use the
\mintinline{R}{c} command to \textit{concatenate} these base structures
together:

\begin{minted}{R}
c(a, 4)
\end{minted}

There are a number of different types of variables in R, here is a very
brief list of some of them:

\begin{itemize}
    \item Integers -- \mintinline{R}{integer} -- for example \mintinline{R}{2},
        \mintinline{R}{4}, \mintinline{R}{-459060}.
    \item Floats -- \mintinline{R}{double} -- for example \mintinline{R}{2.0},
        \mintinline{R}{3.4}, \mintinline{R}{-3.459060}.
    \item Strings -- \mintinline{R}{character} -- for example
        \mintinline{R}{"two"}, \mintinline{R}{"hello world"}, \mintinline{R}{"3450"}.
    \item Booleans -- \mintinline{R}{logical} -- for example
        \mintinline{R}{TRUE} or
        \mintinline{R}{FALSE}.
\end{itemize}

Based on the values of a variable it is possible to construct Booleans:

\begin{minted}{R}
is_a_larger_than_b <- a > b
\end{minted}

The variable \mintinline{R}{is_a_larger_than_b} will be the boolean variable
\mintinline{R}{FALSE}.

This is an important concept as boolean variable allow us to use conditional
statements that let us write code that does specific things based on the value
of variables. For example the following code will add 5 to the smallest
variable:

\begin{minted}{R}
a <- 3
b <- 5
if (a < b) {
  a <- a + 3
} else if (a > b) {
  b <- b + 3
} else {
  a <- a + 3
  b <- b + 3
}
\end{minted}

If you are experimenting by typing the code as you go change the value of
\mintinline{R}{a} or \mintinline{R}{b} to see how the behaviour changes.
What happens if they are equal?

R is a so called ``vectorized'' language which means that there is often a more
appropriate approach to doing things repeatedly using vectors. This applies to
the \mintinline{R}{if} statement in that there exists a \mintinline{R}{ifelse}
statement that applies to vectors of booleans. For example:

\begin{minted}{R}
booleans <- c(FALSE, TRUE, FALSE, FALSE)
ifelse(booleans, "cat", "dog")
\end{minted}

It is also possible to use conditional statements to repeat code. For
example the following code will repeatedly add 1 to the smallest variable until
it becomes equal to the largest one:

\begin{minted}{R}
a <- 3
b <- 5
while (a != b) {
  if (a < b) {
    a <- a + 1
  }
  else {
    b <- b + 1
  }
}
\end{minted}

It is important to be able to reuse code, this is done using a programming
concept called a \textit{function}, which acts similarly to a mathematical
function.

The following code, creates a function that takes two variables as input and
outputs the largest number and the smallest increased by 3.

\begin{minted}{R}
add_3_to_smallest <- function(a, b) {
  # This function adds 3 to the smallest of a or b.
  if (a < b) {
    return(c(a + 3, b))
  }
  else {
    return(c(a, b + 3))
  }
}
\end{minted}

Note that R will implicitly return the last computed expression without the need
for a \mintinline{R}{return} statement. So the above can also be written as:

\begin{minted}{R}
add_3_to_smallest <- function(a, b) {
  # This function adds 3 to the smallest of a or b.
  if (a < b) {
    c(a + 3, b)
  }
  else {
    c(a, b + 3)
  }
}
\end{minted}

Once we have defined the function, the following is how we use it:

\begin{minted}{R}
add_3_to_smallest(a=5, b=-42)
\end{minted}

It is possible to iterate over elements inside R vectors:

\begin{minted}{R}
tennis_players <- c("Federer", "S. Williams", "V. Williams", "Nadal", "King")
\end{minted}

The following will print all the names contained in the vector:

\begin{minted}{R}
for (name in tennis_players) {
    print(name)
}
\end{minted}

We will often want to iterate over a vector of integers, R has a \mintinline{R}
{seq}
command that can create such a vector with ease. The following
code will print every 3 integers from 30 to 50:


\begin{minted}{R}
for (i in seq(30, 50, 3)) {
  print(i)
}
\end{minted}

A final important aspect of R is that of packages. The code examples above
are from the so called `base R' but R has numerous packages
specific to given problems. If you want to download and use one you can always
do so as long as you have an internet connection.

For example, to download a very common collection of data science tools called
\mintinline{R}{tidyverse}
we use the following line of code inside of an R session:

\begin{minted}{R}
install.packages{"tidyverse")
\end{minted}

Once this package is installed you load it using

\begin{minted}{R}
library(tidyverse)
\end{minted}
