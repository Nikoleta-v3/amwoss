\chapter[Discrete Event Simulation]{Discrete Event Simulation}

% Introduction
\chapterinitial{C}{omplex} situations further compounded by randomness appear
throughout our daily lives. For example, data flowing through a computer
network, patients being treated at an emergency services, and daily commutes to
work.
Mathematics can be used to understand these complex situations so as to
make predications which in turn can be used to make improvements. One tool used
to do this is to let a computer create a dynamic virtual representation of the
scenario in question, the particular type we are going to cover here is called
Discrete Event Simulation.

\section{Problem}\label{sec:problem}

Consider the following situation: a bicycle repair shop would like reconfigure
their set-up in order to guarantee that all bicycles processed by the repair
shop take a maximum of 30 minutes.
Their current set-up is as follows:

\begin{itemize}
  \item Bicycles arrive randomly at the shop at a rate of 10 per hour.
  \item They wait in line to be seen at an inspection counter, manned by one
  member of staff who can inspect one bicycle at a time. On average an
  inspection takes around 3 minutes.
  \item After inspection it is found that around 20\% of bicycles do not need
  repair, and they are then ready for collection.
  \item After inspection is is found that around 80\% of bicycles go on to be
  repaired. These then wait in line outside the repair workshop, which is manned
  by two members of staff who can each repair one bicycle at a time. On average
  a repair takes around 6 minutes.
  \item After repair the bicycles are ready for collection.
\end{itemize}

A diagram of the system is shown below:

We can also assume that there is infinite capacity at the bicycle repair shop
for waiting bicycles.
The shop will hire and extra member of staff in order to meet their target of a
maximum throughout of 30 minutes. They would like to know if they should work on
the inspection counter or in the repair workshop?


\section{Theory}\label{sec:theory}

A number of the events that govern the behaviour of the bicycle shop above are
probabilistic, or act probabilistic under a lack of information. For example the
times that bicycles arrive at the shop, the duration of the inspection and
repairs, and whether the bicycle would need to go on to be repaired or not.
When a number of these probabilistic events are arranged in a complex system
such as the bicycle shop, using analytical methods to manipulate these
probabilities becomes difficult. One methods to deal with this is
\textit{simulation}.

Consider one probabilistic event, rolling a die.
A die has six sides numbered 1 to 6, each side is equally likely to land.
Therefore the probability of rolling a 1 is $\frac{1}{6}$, the probability of
rolling a 2 is $\frac{1}{6}$, and so on. This means that that if we roll the die
a large number of times, we would except $\frac{1}{6}$ of those rolls to be a 1.
This is called the \textit{law of large numbers}.

Now imagine we have an event in which we do not know the analytical probability
of it occurring. This could be due to lack of information or because the event
is the result of a complex system. Consider rolling a weighted die, that is a
die in which the probability of obtaining one number is much greater than the
others. How can we estimate the probability of obtaining a 5 on this die?

Rolling the weighted die once does not give us much information.
However due to the law of large numbers, we can roll this die a number of times,
and find the proportion of those rolls which gave a 4. The more times we roll
the die, the closer this proportion approaches the underlying probability of
obtaining a 4.

For a complex system such as the bicycle shop, we would like to estimate the
proportion of bicycles that take longer than 30 minutes to be processed. As it
is a complex system it is difficult to work this out analytically. So we would
like to `run' this system a number of times and record the overall proportions
of bicycles spending longer than 30 minutes in the shop.
However unlike rolling a weighted die, it it costly to observe this shop over a
number of days with identical conditions. In this case it is costly in terms of
time, as the repair shop already exists. However some scenarios, for example the
scenario where the repair shop hires and additional member of staff, do not yet
exist, so observing this this would be costly in terms of money also.
We can however build a virtual representation of this complex system on a
computer, and `run' a day of work much more quickly and much less costly on the
computer, similar to a video game.

% Computers not good at generating random numbers, so use pseudo random numbers. e.g. (formula)
In order to do this, the computer needs to be able to generate random outcomes
of each of the smaller events that make up the large complex system. Generating
random events are essentially functions of random numbers, that need to be
generated.

Computers are not good at generating true randomness.

They can however generate pseudorandom numbers: sequences of numbers that look
random, and behave as random numbers, but are entirely determined from the
previous number in the sequence. Every programming language has methods of doing
this.

In order to simulate an event we can again manipulate the law of large numbers.
Let $X \sim U(0, 1)$, a uniformly pseudorandom variable between 0 and 1.
Let $R$ be the outcome of a roll of an unbiased die. Then $R$ can be defined as:

\begin{equation}
R =
  \begin{cases}
    1 & \text{if } 0 \leq X < \frac{1}{6}\\
    2 & \text{if } \frac{1}{6} \leq X < \frac{2}{6}\\
    3 & \text{if } \frac{2}{6} \leq X < \frac{3}{6}\\
    4 & \text{if } \frac{3}{6} \leq X < \frac{4}{6}\\
    5 & \text{if } \frac{4}{6} \leq X < \frac{5}{6}\\
    6 & \text{if } \frac{5}{6} \leq X < 1
  \end{cases}
\end{equation}

% What we gonna do: use pseudo random numbers to 'simulate' the system. - because law of large number we can simulate many times and smoothe.
The bicycle repair shop is a system made up of interactions of a number of other
simpler random events. This can be thought of a functions of functions of
functions of random variables, each generated using pseudorandom numbers.

In this case the random simpler random event that need to be generated are:
\begin{itemize}
  \item time each bicycle arrives to the repair shop,
  \item the time each bicycle spends at the inspection counter,
  \item whether each bicycle needs to go on the the repair workshop,
  \item the time each those bicycles spends at the repair shop.
\end{itemize}

As the simulation progresses these events will be generated, and will interact
together as described in Section~\ref{sec:problem}.
The proportion of customers spending longer than 30 minutes in the shop can then
be counted. This proportion itself is a random variable, and so just like the
weighted die, running this simulation once does not give us much information.
But we can run the simulation many times and take an average proportion, so
smooth out any variability.

% Two types of DES - event scheduleing in Python, process-based in R.
The process outlined above is called \textit{discrete event simulation}, a
particular implementation of Monte Carlo simulation, which generates
pseudorandom numbers and observes their interactions. In practice there are two
main approaches to simulating complex probabilistic systems such as this one:
the \textit{event scheduling} approach, and \textit{process based} simulation.
It just so happens that the main implementations in Python and R use each of
these approaches.

% Event scheduling
\subsection{Event Scheduling Approach}

% Process based
\subsection{Process Based Simulation}


\section{Solving with Python}\label{sec:solving-with-python}

- Build ciw model with some set of parameters
- Run ciw model (with trials) and find that you can't do it.
- Write function to build ciw model with given set of parameters
- Loops.
- Conclusion? (possibly visualisation)


\section{Solving with R}\label{sec:solving-with-R}

- Build simmer model with some set of parameters
- Run simmer model (with trials) and find that you can't do it.
- Write function to build simmer model with given set of parameters
- Loops.
- Conclusion? (possibly visualisation)

\section{Research}\label{sec:research}

TBA
